\documentclass[12pt]{article}

\usepackage{amsmath}
\usepackage{amssymb}
\usepackage{graphicx}

\oddsidemargin-1cm
\topmargin-2cm
\textwidth18.5cm
\textheight23.5cm  
 
\title{P3 Proposal: ByteTorrent}
\author{Connor Brem\\Billy Wood}
\date{April 15, 2014}

\begin{document}

\maketitle

\section*{Description}

ByteTorrent will be a BitTorrent-esque file-sharing application.
To run ByteTorrent, we will implement two types of nodes: Clients and Trackers.
As with BitTorrent, Clients will share files chunk-by-chunk. Trackers will
inform clients which other clients possess chunks of that file.

In order to increase reliability, Trackers will be implemented as clusters,
which use Paxos for replication. A Tracker cluster should always be ``up''.

Clients, on the other hand, can go on- and off-line as they please without
losing the ``right'' to serve file chunks to other Clients when they are online.
To ensure this behavior, one Client will never ask a Tracker to remove another
unresponsive Client from the Tracker's record. Instead, if one Client thinks
that another Client is unresponsive, the first Client will back off.

We ultimately intend to implement graphical interfaces for both
the Client and the Tracker.
Users will be able to use the Client's interface to manage downloads in
progress and construct .torrent files. They will be able to use the Tracker's
interface to upload or browse for .torrent files.


\section*{Structure}

\subsection*{Client}
The Client nodes will be implemented in Go.
They will have the following abilities:
\begin{itemize}
\item  Create new .torrent files and send them to a Tracker
\item  Given a .torrent file, connect to a Tracker identified in the
       .torrent file to determine which other Clients hold chunks of the
       corresponding data file and fetch these chunks.
\item  Serve chunks of a file to other Clients on request
\item  Inform a Tracker when they have a chunk of a file
\item  Report themselves when they do \emph{not} have a chunk that other clients
       think they do
\item  Detect when a Tracker node with which this client is communicating
       becomes unresponsive, and resume communication with the new master in
       this Tracker's cluster.
\item  Expose some of these functions (create .torrent files, fetch file chunks)
       through a GUI. This GUI will be implemented in Go (using go-gtk:
       mattn.github.io/go-gtk), since the client UI should be on the same
       machine as the rest of the client.
\end{itemize}

\subsection*{Tracker Server}
The Tracker nodes will need to be highly distributed.
As such, we've decided to implement Trackers in Go.
The Trackers will be able to:
\begin{itemize}
\item  Receive new .torrent files from Clients
\item  Send existing .torrent files to Clients
\item  Keep record of which Clients have which chunks of files
\item  Maintain state within a cluster despite node failures (via Paxos).
\item  Expose some of these functions (add .torrent files, fetch/browser
       .torrent files) via a GUI. This GUI will be implemented in
       Javascript/HTML, since it has low network requirements and users
       will likely not be on the same machine as the tracker.
\end{itemize}


\section*{Distributed Systems Concepts}

\begin{itemize}
\item  \textbf{Exponential Backoff:} The Client nodes will be in constant
       contact with each other.
       They will need to be robust to neighbors dying, and Trackers dying.
       However, they must also be able to move on- and off-line without other
       clients reporting them as unresponsive.
       We plan to use exponential back-off to disengage from unresponsive
       connections.
\item  \textbf{Paxos:} Tracker servers will be organized into clusters.
       Within any cluster, all data will be replicated with Paxos.
\item  \textbf{Hashing for integrity:} Torrent files will include hashes of
       the related data files so that clients can check the data files'
       integrity.
\item  \textbf{Hashing for deduplication:} Trackers will compare the hashes
       included with .torret files to identify different .torrents that refer
       to the same content.
\item  \textbf{RPC:} The Clients will make Remote Procedure Calls to the
       Trackers in order to get certain pieces of data.
\end{itemize}

\section*{Test Plan}

\subsection*{Tracker Tests}
\begin{itemize}
\item  Kill a Tracker node in a Tracker cluster,
       and make sure that the affected Tracker cluster is still useful.
\item  Kill sufficiently many Trackers to stop progress,
       and make sure that all progress halts.
\item  Create slow Trackers,
       and make sure that the Tracker cluster stays consistent.
\end{itemize}

\subsection*{Client Tests}
\begin{itemize}
\item  Kill a peer which has a chunk that Client $A$ wants, and ensure that
       client $A$ backs off and retrieves the chunk from another peer.
\item  If some Tracker in a Tracker cluster becomes unresponsive, ensure that
       Clients find and resume communication with the new leader in the cluster.
\item  Ask a Client for a chunk that it no longer has, and ensure that the asked
       Client reports that it does not have the chunk.
\item  Test with a slow Tracker cluster on start-up.
\end{itemize}

\section*{Development Tiers}

\subsection*{Minimum Testable Product}
Our primary goal is to create a Paxos-replicated Tracker.
As such, our minimum product will have most of the same Tracker functionality,
with a simplified Client program.
In addition, our minimum product will only support a single .torrent file per
Tracker cluster.

\begin{itemize}
\item  Client:
    \begin{itemize}
    \item  Get chunks from other clients
    \item  Serve chunks to other clients
    \item  Can fail without disrupting things
    \item  Cannot create new .torrent files.
    \item  Will not report when Client does not have a chunk
    \end{itemize}
\item  Tracker:
    \begin{itemize}
    \item  Maintains data for a single .torrent file
    \item  Paxos-replicated cluster
    \item  Record when new clients get chunks of files
    \item  Does not need to receive new .torrent files
    \end{itemize}
\end{itemize}

\subsection*{Target Deliverable}
Our target deliverable will allow for multiple .torrent files to exist on
each Tracker cluster.
At this stage, we also make the Client program slightly more complicated.

\begin{itemize}
\item  Client:
    \begin{itemize}
    \item  Self-report to tracker when Client does not have a chunk that other
           Clients expect it to
    \item  Compute/create checksums on torrents to ensure security
    \item  Maintain target seed-to-leech ratio
    \end{itemize}
\item  Tracker:
    \begin{itemize}
    \item  Record new .torrent files
    \end{itemize}
\end{itemize}

\subsection*{Stretch Goal}
For our final development stage, we want to add user interfaces to the Client
and Tracker, for ease of use.

\begin{itemize}
\item  Client:
    \begin{itemize}
    \item  User Interface
    \end{itemize}
\item  Tracker:
    \begin{itemize}
    \item  Deduplicate distinct .torrents which refer to the same content
    \item  Maintain users/manage logins
    \item  User Interface
    \end{itemize}
\end{itemize}

\section*{Development Schedule}

\subsection*{Week 1: April 14-20}
For Week 1, our goal will be to have the Minimum Testable Product prepared.
We will primarily spend our time writing a Paxos-replicated Tracker cluster,
and allowing Clients to send and receive chunks.

\subsection*{Week 2: April 21-28}
At the beginning of Week 2, we will focus on getting our project to the Minimum
Testable Product, with working Paxos-replication, in time for the first
code review.

From that point forward, we will work on getting our project to the
Target Deliverable state.
For this, we will mostly be working on the Client, to implement various security
and consistency features.
The Tracker will need minor updates to allow the receipt of new .torrent files.

\subsection*{Week 3: April 29-May 1}
In Week 3, we will work on the Stretch Goals.
This primarily means implementing GUIs for the Tracker and Client,
but we will also need to maintain users and manage logins.

\end{document}
